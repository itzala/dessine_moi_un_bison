\documentclass[a4paper, 12pt]{report}
 
\usepackage[utf8]{inputenc}
\usepackage[T1]{fontenc}
\usepackage[francais]{babel}
\usepackage[top=2cm, bottom=2cm, left=4cm, right=2cm]{geometry}


\title{Rapport de projet : Dessine moi un Bison}
\author{Nicolas \bsc{Endredi}, Baptiste \bsc{Oruezabal}}
\date{\today}


\begin{document}

\maketitle


\setcounter{tocdepth}{3}
\tableofcontents

\newpage
\section{Présentation du projet}
\subsection{Analyse lexicale et syntaxique}
Ecrire un programme implique une vérification de la syntaxe afin de s'assurer que les instructions soient correctement comprises. Tout d'abord, il faut vérifier que les mots sont bien reconnus. Il s'agit de l'analyse lexicale. Ensuite, on regarde que les phrases soient s'enchainent correctement. Nous réalisons l'analyse syntaxique. \\

\subsection{Sujet du projet}
Le but du projet est d'écrire un analyseur lexical (Flex) et syntaxique (Bison) pour reconnaitre un langage et faire du dessin vectoriel.

\newpage
\section{Fonctionnalités}

\newpage
\section{Problèmes rencontrés}

\section{Conclusion}

\end{document}